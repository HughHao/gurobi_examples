\documentclass{article}
\author{Haohaiqiang}
\title{Minimization of production switching cost} % 切换成本最小
\usepackage{enumerate}
\usepackage{amsmath}
\newcommand{\myref}[1]{Eq.(\ref{#1})}
\usepackage{booktabs} % 三线表需要的格式

\begin{document}
	\maketitle
	\section{Scenario of production switching}
	\begin{enumerate}[(1)]
		\item There are two lines namely $ L_1 $ and $ L_2 $, and the production capacities are respectively 24 and 24.\\
		\item Four kinds  of products $ A_1, A_2, B_1 and B_2 $are to be produced, the needs are 14, 10, 12, and 12 respectively. \\
		\item The switching cost between two products is shown in Table \ref{sc}\\
		\item Initial condition: the last state of the previous schedule is $ L_1:A_1 $ and $ L_2:A2 $.
	\end{enumerate}
	\begin{table}
		\centering
		\caption{Switching cost between two products}
		\begin{tabular}{*{5}{c}} 
		\toprule
		& A1 & A2 & B1 & B2 \\
		\midrule
		A1& 0 & 1 & 4 & 4 \\
		
		A2& 1 & 0 & 4 & 4 \\
		
		B1& 4 & 4 & 0 & 1 \\
		
		B2& 4 & 4 & 1 & 0 \\
		\bottomrule
		\label{sc}
		\end{tabular}
	\end{table}
	Objective: to allocate the products into the optimal lines to minimize the switching cost. The mathematical model of the production planning can be described as \myref{sm}\\ The parameter is the number of product kinds $ P $, the number of each kind $ P_i $, the number of lines $ L $, the switching cost $ C_{j,i} $ of product $ i $ that produced immediately after $ j $.
	The relative variables are: $ N_i $ to represent the number of product $ i $ that allocated to line 1. $ X_{j,i} $ to represent whether product $ i $ is produced immediately after $ j $.
	\begin{align}
		\min f &= \min \sum_{i=1}^{P}\sum_{j=1}^{P}(N_i\cdot X_{j,i}\cdot C_{j,i})\\&+\sum_{i=1}^{P}\sum_{j=1}^{P}((P_i-N_i)\cdot X_{j,i}\cdot C_{j,i})\\
		\text{s.t.} \nonumber\\
		&N_i \leq P_i\\
		&\sum_{j=1}^{P}X_{j,i}=1\\
		&\sum_{i=1}^{P}X_{j,i}=1\\
		X_{j,i}&=\begin{cases}
			1, & \text{product $ i $ is produced immediately after $ j $}\\
			0, &\text{otherwise};
		\end{cases}
	\end{align}

	\section{Assembly planning}
	\subsection{Scenario}
	\begin{enumerate}[(1)]
		\item Orders for multi-customers:$ O_1 $,$ O_2 $,$ O_3 $,$ O_4 $.\\
		\item Each order is accompanied with three operations [1,2,3], the processing times are different, as shown in table \ref{order_operations}.\\
		\item The operating resources are $ R_1 $, $ R2 $, and the molds are $ M_1 $ and $ M_2 $.\\
		\item The processing resources and molds required for each operation are also different. This example assumes that each process can start only when a resource and a mold are available.\\
		\item Multiple choices of resources and molds are allowed for each process. For example, processing resources are shown in Table \ref{resources_ope}.\\
	\end{enumerate}
	\begin{table}
		\centering
		\caption{Operating times of each order}
		\begin{tabular}{*{4}{c}}
			\toprule
			& Operation1 & operation2 & Operation3\\
			\midrule
			O1&10&20&30\\
			O2&40&30&20\\
			O3&20&20&30\\
			O4&40&40&30\\
			\bottomrule
			\label{order_operations}
		\end{tabular}
	\end{table}
	\begin{table} 
		\centering
		\caption{Resources for operations}
		\begin{tabular} {*{4}{c}}
			\toprule
			& Operation & Resource & Available\\
			\midrule
			O1&1&R1&Y\\
			O1&1&R2&Y\\
			O1&2&R1&Y\\
			O1&2&R2&N\\
			\bottomrule
			\label{resources_ope}
		\end{tabular}
	\end{table}
	Target: to minimize the completion time of all the orders by maximizing the resources and molds.
	\subsection{Model of the completion times}
	The given parameters are:
	\begin{enumerate}[i.]
		\item the number of orders $ No $;\\
		\item the number of operations of each order $ N_i $;\\
		\item the machining times $ T_{i,j,k,l} $ of operation $ i $ of order $ j $ in machine $ k $ with mold $ l $;\\
		\item $ X_{i,j,k} $whether the operation $ i $ of order $ j $ is available in machine $ k $;\\
		\item $ Y_{i,j,l} $whether the operation $ i $ of order $ j $ is available with mold $ l $;\\
	\end{enumerate}
	\begin{align}
		\min C_{max} &= \min \max\left\{C_{i,j}\right\} \label{obj}\\
		\text{s.t.} \nonumber \\
		C_{i,j} &\geq \max \left\{C_{i-1,j}, F_k, E_l\right\} + T_{i,j,k,l}\cdot X_{i,j,k}\cdot Y_{i,j,l}\\
		C_{0,0}&=0\\
		X_{i,j,k}&=\begin{cases}
			1, &\text{if the machine $ k $ is available for operation $ i $ of order $ j $};\\
			0,&\text{otherwise}.
		\end{cases}\\
		Y_{i,j,l}&=\begin{cases}
			1, &\text{if mold $ l $ is available for operation $ i $ of order $ j $};\\
			0,&\text{otherwise}.
		\end{cases}
	\end{align}
\end{document}